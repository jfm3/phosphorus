% NOTE TO TYPESETTERS: Two somewhat obscure yet prideful points to
% keep in mind about this document.  (1) Words should not be
% hyphenated.  In no case should hyphens be allowed into the document,
% or you'll ruin the joke.  (2) The footnotes must remain
% typographically intrusive and ugly, as it reinforces the badness of
% named parenthetical things.  Please do not relegate them to a
% separate page.

% Can I just say that I fucking love eplain?
\input eplain

% magnifying small fonts gives titles an ironic oldSchool flair
\font\titlefont=cmr5 at 10pt

% plain TeX's \beginsection is too eager to make column breaks
\def\section#1\par{\bigskip\noindent{\bf #1}\nobreak\par\noindent}

% I fucking love eplain.
\let\footnote = \numberedfootnote

\topmargin = 1in
\bottommargin = 1in
\leftmargin = 1.9in
\rightmargin = 1.9in
\paperheight = 11truein
\paperwidth = 8.5truein

\clubpenalty=300
\widowpenalty=300

% Part of the humor in this paper are the freshman and awful abuses of
% TeX.  The following effects the main joke, by castrating TeX's
% incredibly sophistcated hyphenation and line breaking algorithms.
\tolerance=10000
\hbadness=10000
\hyphenchar\font=-1
{\tt \hyphenchar\font=-1}
{\it \hyphenchar\font=-1}
{\titlefont \hyphenchar\font=-1} % can't be too careful

\def\bull#1{\medskip\hbox{\vtop{\hsize12pt{$\bullet$\hfil}}
\vtop{\hsize4.5in{#1}}\hfil}}

\centerline{\titlefont phosphoCon 2011 callForPapers}
\smallskip
\centerline{{\it Vae victis!}}
\medskip

PhosphoCon continues to be the world's best resource for learning and
sharing about the Phosphorous programming language.  PhosphoCon 2011
shall continue in the tradition of singleTrack ``bigRoom'' style
presentations, punctuated by informal breakout sessions where it is
{\it totally cool} to hand out your moo cards to every obscenely
wealthy 20Something you can find.

PhosphoCon welcomes research articles from scientists, Engineers,
researchScholars, and practitioners of the Art of computerProgramming,
across any academic field or application domain, and from all over the
world.  Papers for publication in PhosphoCon proceedings are selected
through vigorous peer review that in no way is influenced by the
things people were complaining about last year.  Be advised that
papers containing the hyphen character or the LWord will be discarded
immediately, and that we prefer papers relating to the following
conference topics.  {\it Ignotium per ignotius!}

\section noL

The addition of the noL module to Phosphoros brings Phosphorous
programmers into the main stream of the noL movement.  This is great
because as even Guy Steele will tell you, programming languages are
not web scale\footnote{{\tt http://tinyurl.com/33ttffw}}.  {\it
  Ientaculum per rex!} Possible paper topics include:

\unorderedlist
\li How to get noL into your shop when your boss says no.
\li Why all that old L crap sucks, and how noL rules.
\li No, seriously dude, it rules.
\endunorderedlist

\section Latin is officiallySexy

By use of an Internet Poll, programmers of Phosphoros have decided
that the Latin language shall be part of our branding and
popularization structure.  Latin is obscure enough so that errors are
not obvious to the casual reader, yet arcane enough to invoke an
unparalleled sense of cleverness about the author.  Phosphoros
therefore aims to be the Latin of programming languages, and as such
we shall use Latin phrases liberally throughout our writing, to remind
the reader of this marketing lever. {\it Quidquid latine dictum sit
  altum videtur!}  Possible paper topics include:

\unorderedlist
\li Essential properties of the category of Resurrected languages.
\li We're running out of Latin phrases already, actually.
\endunorderedlist

\section bizDevs, Seth Godin, and Popularity

It is a commonly accepted fact that Java(TM) became popular not
because it is intrinsically good, but because Sun spent millions of
dollars on {\it marketing} to convince companies to force their C++
developers to use it.  The best marketing expert of our age is Seth
Godin, mostly because he has a blog and uses only words we understand.
{\it Quis custodiet ipsos custodes!}  Possible paper topics include:

\unorderedlist
\li {\it All Marketers are Liars.}  Are all PhosphoDevs liars?  Should
we call them ``stories'' instead of programs?  WWSGD?
\li Is our meatball sundae a free prize that wins us permission from
our tribe?
\endunorderedlist

\bigskip
\bigskip

\section phosphorusMT

Machine translation (MT) is a technique used by computational
linguists to translate from one natural language to another.  High
degrees of success have been obtained in MT by ignoring deep
structural analysis of language and instead performing statistical
analyses of word frequency and position correlation between the source
and target languages.  The Phosphorous community believes that similar
techniques can be used to compile programs, such that even incorrect
programs will be ``translated'' into the desired object code.  {\it
  Hesperus Phosphorus est!}  Possible paper topics include:

\unorderedlist
\li Will MT of programs lead to a new definition of Kolmogorov
complexity?  (Here is your chance to use displayed math and other
special typography in \TeX.)
\li Can one write probabilistic algorithms by introducing explicit bugs in
nonProbabilistic algorithms and compiling {\tt +mt}?
\endunorderedlist

\section idiomaticPhosphorus

Modern programming languages have grown in complexity to the point
where instead of just knowing a base language and gaining familiarity
with a set of libraries, one also must learn a set of {\it idioms} for
that language.  Lack of idiomatic knowledge means a programmer will
miss out on certain features, produce suboptimal code, and intoduce
maintenance problems.  This has led to the exciting new field of idiom
design.  {\it Mutatis mutandis!}  Possible paper topics include:

\unorderedlist
\li Language design is ... idiom design?
\li Is it better for the need for an idiom to be a complex concept, or
a workaround to shoehorn a useful notion into the rigid formalisms of
a language?  Will Haskell win the idiom war?  Will Python?
\endunorderedlist

\section syntacticExtension License Examination Questions

As always, syntactic extension of the Phosphorus language may only be
performed by licensed and bonded experts.  Licensing examinations will
be provided at the conference, and new MacroWritingRightsManagement
keys will be distributed.  We solicit short papers suggesting new
questions for the exams.  {\it Solvitur ambulando!}

\section thunderTalks

A hugely popular segment of our informal phosphoMeetUps has been
``thunderTalks'', a five minute talk in which you yell and rant about
your pet programming language issues and pound your fist on a lectern.
Cheap beer will be available.  {\it Potior est, qui prior est!}
\bigskip
{\center
{\titlefont Submissions Due By Feb 28 2011}\footnote{Unless your adviser was in our fraternity.}

{\tt pc11Submissions@nectarine-city.com}
}
\bigskip
% damnit, which font is this anyway? sectionfont or something?
%\section {\hyphenchar\font=-1 WAXY WHITE LEVEL SPONSORS}

%\unorderedlist
%\li Nectarine City, LLC ({\tt http://nectarine-city.com})
%\endunorderedlist

\bye
