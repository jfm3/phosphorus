% NOTE TO TYPESETTERS: Two somewhat obscure yet prideful points to
% keep in mind about this document.  (1) Words should not be
% hyphenated.  In no case should hyphens be allowed into the document,
% or you'll ruin the joke.  (2) The footnotes must remain
% typographically intrusive and ugly, as it reinforces the badness of
% named parenthetical things.  Please do not relegate them to a
% separate page.

% Can I just say that I fucking love eplain?
\input eplain

% magnifying small fonts gives titles an ironic old-school flair
\font\titlefont=cmr5 at 10pt

% plain TeX's \beginsection is too eager to make column breaks
\def\section#1\par{\bigskip\noindent{\bf #1}\nobreak\par\noindent}

% I fucking love eplain.
\let\footnote = \numberedfootnote

\topmargin = 1in
\bottommargin = 1in
\leftmargin = 1.9in
\rightmargin = 1.9in
\paperheight = 11truein
\paperwidth = 8.5truein

% Part of the humor in this paper are the freshman and awful abuses of
% TeX.  The following effects the main joke, by castrating TeX's
% incredibly sophistcated hyphenation and line breaking algorithms.
\tolerance=10000
\hbadness=10000
\hyphenchar\font=-1
{\tt \hyphenchar\font=-1}
{\it \hyphenchar\font=-1}
{\titlefont \hyphenchar\font=-1} % can't be too careful

\def\bull#1{\medskip\hbox{\vtop{\hsize12pt{$\bullet$\hfil}}
\vtop{\hsize4.5in{#1}}\hfil}}

\centerline{\titlefont phosphoConf 2010 callForPapers}
\smallskip
\centerline{{\it The Next Fifty Weeks}}

%\begin{verbatim}
%phs> |10 qlS 
%         |20 qt Popular 2|
%         |30 qt Better\\! 3|1|
%tr
%\end{verbatim}

\section twitterateProgramming

As the programming language Fortress has taught us, programming
languages will be more popular if they present code in a way
programmers like to look at (scientists all love to look at
\TeX\ output, for example).  After extensive human factors studies,
PhosphoDevs determined that 140 character ``tweets'' were the most
modern and popular presentation mechanism for textual information.
Indeed, the restriction to 140 characters makes users compress their
messages into semiProgrammatic shortHand representations anyway.
Possible paper topics include:

\unorderedlist
\li Advantages of using Tinyurl over Github.
\li The tweetEvalTweetLoop: a great thing, or the
{\bf best} {\bf thing} {\bf ever}?
\li Efficient algorithms for Tweave and Twangle.
\endunorderedlist

\section kindedPhosphorus

Functional programming languages such as Haskell and OCaml claim to
have eliminated large classes of programmer errors by imposing strict
type systems.  Exciting forays into this area have been made with the
release of kindedPhosphorus, an extension of the base Phosphorus with
{\it kinds}, which are far more complicated, restrictive, and
bugEliminating than mere types.  We expect many exciting papers that
use Greek letters, exotic typography, and other impressiveLooking
features of \TeX\ to result from this work, such as the following.

\unorderedlist
\li Does limiting the number of useful programs programmers can write
really count as ``fewer bugs''?
%\li Backward compatibility with the (now deprecated)
%unicornTyping\footnote{Human factors studies showed users liked
%  unicorns more than ducks.} system.
\li What should we use {\tt Fraktur} for?
\endunorderedlist

\section boltOriented Programming

Long strides have been made in the past fifty weeks in the field of
boltOriented programming, a programming discipline which Phosphorus
excels at.  Paper topics include:

\unorderedlist
\li The boltOriented community and the kindedPhosphorus communities hate
  each other.  How can these two subsystems be made more incompatible?
\li Novel ideas about the metaBolt protocol and other threadStripping
  techniques.  Efficient strategies for compiling cageNuts.
\li Critiques of the ``gang of several'' erectorSet theory and the
``quality that can not be screwed.''
\endunorderedlist

\section visualPhosphorusForAccountants And Deprecating phosphArc 

Elimination of needless parenthesis from multiIf (similar to Lisp
COND) has met a cold reception by the Phosphorus community.  Arc is
simply not a popular language.

On the other hand, one of the implementers of the VBA language, Joel
Spolsky, is hugely popular.  phosphoDevs are often heard repeating the
clever epithet ``bend it like Spolsky'', which refers to his 2006
essay ``Language Wars''\footnote{{\tt
http://www.joelonsoftware.com/items/2006/09/01.html}} in which he
simultaneously proclaims that languages must have a ``gigantic
ecosystem'' to be viable for ``web development'', and that his
company's flagship software offering is written in Wasabi, a language
they invented for themselves for ``web development'' that resembles
Visual Basic, but has metaprogramming and code generation
features.\footnote{{\tt
http://www.fogcreek.com/FogBugz/blog/category/Wasabi.aspx}.  N.B. The
``gigantic ecosystem'' for Wasabi is never described.}  After thus
debating the language issue, he proclaims that ``debating the merits
of programming languages [is] a fruitless debate if I've ever seen
one.''  This ability to take both sides of an issue and make it appear
both coherent and pointless at the same time is something we view as
critical for our ``popular is better'' approach.

\unorderedlist
\li How can we make Phosphorus less like Arc and more like VBA?  
\li It has been proposed that Popularization techniques should pass 
a ``Spolsky test'' to determine if they work or not.  What are some
  good Spolsky tests?  For example; does a given PopTech work to
  popularize the idea that a Lisp-3 is {\it one better} than a Lisp-2?
\li How can we distance ourselves from PG without losing the attention
  of small startUp companies that want to emulate Viaweb's success?
\endunorderedlist

%% \section Vampires

%% Vampires are hugely popular.  

%% \unorderedlist
%% \li How can we work them into Phosphorus?
%% \endunorderedlist

\section starPhorus

We are basically obliged to accept any paper on how to make a
programming language better for parallel programming, mostly because
the best ideas so far are actors and STM, which are both incredibly
lame.  

\unorderedlist
\li The most popular solutions to the parallel programming problem are
language features boltedOn, such as the CUDA SDK.  Are Phosphorus'
boltOriented language features a good fit for such solutions?
\li I want to put a joke about the Python GIL here, but I can't.  {\it
  It's just too painful.}
\endunorderedlist

\section syntacticExtension License Examination Questions

As always, syntactic extension of the Phosphorus language may only be
performed by licensed and bonded experts.  Licensing examinations will
be provided at the conference, and new MacroWritingRightsManagement
keys will be distributed.  We solicit short papers suggesting new
questions for the exams.

\section thunderTalks

A hugely popular segment of our informal phosphoMeetUps has been
``thunderTalks'', a five minute talk in which you yell and rant about
your pet programming language issues and pound your fist on a lectern.
Cheap beer will be available.  First come first serve.
\bigskip
{\center
{\titlefont Submissions Due By Jan 12 2010}\footnote{Unless your adviser was in our fraternity.}

{\tt pc09-submissions@nectarine-city.com}
}
\bigskip
% damnit, which font is this anyway? sectionfont or something?
\section {\hyphenchar\font=-1 WAXY WHITE LEVEL SPONSORS}

\unorderedlist
\li Nectarine City, LLC {\tt http://nectarine-city.com}
%\li Steel Bank Studio {\tt http://sb-studio.net}
%\li Gigamonkeys Consulting {\tt http://gigamonkeys.com}
\endunorderedlist

\bye
