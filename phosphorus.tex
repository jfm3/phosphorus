% NOTE TO TYPESETTERS: Two somewhat obscure yet prideful points to
% keep in mind about this document.  (1) Words should not be
% hyphenated.  In no case should hyphens be allowed into the document,
% or you'll ruin the joke.  (2) The footnotes must remain
% typographically intrusive and ugly, as it reinforces the badness of
% named parenthetical things.  Please do not relegate them to a
% separate page.

% Can I just say that I fucking love eplain?
\input eplain

% magnifying small fonts gives titles an ironic old-school flair
\font\titlefont=cmr5 at 10pt

% plain TeX's \beginsection is too eager to make column breaks
\def\section#1\par{\bigskip\noindent{\bf #1}\nobreak\par}

% I fucking love eplain.
\let\footnote = \numberedfootnote

\topmargin = .5in
\bottommargin = .5in
\leftmargin = .5in
\rightmargin = .5in
\paperheight = 11truein
\paperwidth = 8.5truein

% Part of the humor in this paper are the freshman and awful abuses of
% TeX.  The following effects the main joke, by castrating TeX's
% incredibly sophistcated hyphenation and line breaking algorithms.
\tolerance=10000
\hbadness=10000
\hyphenchar\font=-1
{\tt \hyphenchar\font=-1}
{\it \hyphenchar\font=-1}
{\titlefont \hyphenchar\font=-1} % can't be too careful

\centerline{\titlefont Phosphorous, The Popular Lisp}
\centerline{Joseph F. Miklojcik III {\tt <jfm3@jfm3.org>}}
\bigskip
\centerline{\bf Abstract}

We present Phosphorous; a programming language that draws on the power
and elegance of traditional Lisps such as Common Lisp and Scheme, yet
which brings those languages into the 21st century by ruthless
application of our ``popular is better'' philosophy into all possible
areas of programming language design.

\doublecolumns

\section Introduction

Modern and popular writing is not prefaced, overviewed, preambled,
foreworded or otherwise introduced.  This is just here to let the
oldTimers know what is happening.  It should also be noted here that,
for similar reasons, there will be no conclusion, epilogue, summary,
or review at the end\footnote{We do not expect the oldTimers will
figure it out, but we didn't want a lot of email from them, hence this
section.}.

\section Name

The name ``Lisp'' is far too tarnished for a popular, modern language.
We determined that there were three characteristics a good modern
programming language name should have.  One, that it begin with the
letter P.  Two, that it name a gemstone.  Three, that it be part of
the name of a popular comedy group.  We therefore named it
``Phosphorus\footnote{We are aware that Phosphorus is not a precious
or semiPrecious stone.  Please stop emailing.}''.

\section Case

Any even somewhat traditional Lisp separates words in names using
hyphens.  Most traditional Lisps have very confusing rules regarding
the case sensitivity of their names.

Modern and popular programming languages have settled on case
sensitive names with words separated by changing case (``CamelCase'').
Phosphorous follows suit.  To underscore this fact, Phosphorous
documentation replaces all uses of the hyphen in English with
CamelCase\footnote{It took some doing, but we also figured out how to
stop \TeX{} from introducing hyphens into our document.  Use the
following.

\verbatim| \tolerance=10000
\hbadness=10000
\hyphenchar\font=-1
{\tt \hyphenchar\font=-1} |endverbatim }.

\section Parenthesis

The most common complaint heard from new Lisp programmers must be ``I
can't stand all these parenthesis.''  At the same time, the success of
XML in the marketplace shows that genericity of syntax is still a
desirable feature.  Clearly, the parenthesis must go, but can we do
better than XML?  The answer is emphatically ``Yes!''

After careful study, we determined the problem stemmed from the fact
that all parenthesis were alike.  A programmer faced with the string
\smallskip
\centerline{\tt <table><tr><td><font=fixed><b>Foo</b>.}
\centerline{\tt <br/></font></tr></td></table>}
\smallskip
\noindent can immediately see the obvious error, whereas in the string
\smallskip
\centerline{\tt ((((((FOO))))))\footnote{\rm This and the preceding
example are actual code taken from production working code at an
Internet startup we cannot name due to {\it legal reasons}.}}
\smallskip
\noindent there are no such visual clues.  In fact XML constructions
such as {\tt <b>} and {\tt </b>} are just {\it named parenthesis}.  We
therefore experimented with similar named parenthesis in early
versions of PhosphorousImpl (AvariciousArmoire\footnote{In keeping
with other popular software projects, versions are named by
adjectiveFurniture pairs, where successive versions (more properly
termed ``releases'') begin with pairs of successive letters of the
English alphabet. We could not come up with the zeroth or negative
first letter of the alphabet, so simply began our release names with
AvariciousArmoire and documented the reason why.}).

Early attempts at named parenthesis implementations were like XML
minus one interior angle bracket from either side, with arbitrary
matching names marking matching angle brackets.  This had the visually
pleasing effect of markers which appeared balanced, like parenthesis,
yet looked enough like XML from a distance\footnote{or squinting} to
fool your programmerManager.  The problem with this approach was that
it was possible to write an Emacs macro to allow the programmer to
simply use angle brackets like parenthesis, labeling them
automatically and hiding the names from view.  We found that to force
programmers to use the named parenthesis syntax, we had to replace
both left and right angle brackets with unmatchable characters, namely
the vertical bar, thus foiling easy Elisp program analysis routines.
To get the nice balance back, we decided that names had to be numbers
that matched with the number that was the same but backwards.  This is
both pleasing to the eye, and really puts the screws to people who
want to thwart the system using Emacs.  To allow Phosphorus code to
continue to have the look and feel of XML, angle brackets and any
characters between them are ignored by the parser.  Here are some
example PhosphoExprs.
\medskip
\centerline{\tt |123 321| <a null list>}
\medskip\nobreak
\centerline{\tt <xmlfk:xm17r>|123 321|</xmlfk:xml17r>}
\centerline{\tt <!-- same, but dressed up to look like xml --!>}
\medskip\nobreak
\centerline{\tt |44 a b c 44| <the list (a b c)>}
\medskip\nobreak
\centerline{\tt |1 anonFunction |2 x 2| |3 numPlus x 1 3| 1|}
\centerline{\tt <like (lambda (x) (+ x 1.0))>}

\section Mathematical Formulas

By typesetting this document in \TeX{}, we are basically obliged to
include some complicated mathematical equations.  It is also a well
known fact that conference papers are scanned for displayed equations,
only those with difficult math then taken seriously enough for primary
consideration.  Therefore, in keeping with our ``popular is better''
philosophy, we hereby popularize this document by reproducing one of
the author's favorite calculus problems from high school.\footnote{It
has been years since I went to high school, and I don't remember much
more than that, other than that the solution has a clever bit of
recursion in it.  Hopefully, this will be enough displayed math.}

$$\int e^{x} \sin{x}\ dx$$

\section Fixing {\tt COND}

As pointed out in the extensive literature published by Paul Graham
[CITATION NEEDED], the often used {\tt COND} Lisp form is too verbose
by two parenthesis per case ($\Theta(2n)$\footnote{The use of the
Master Theorem here allows us to win a bet that I would one day use it
for something other than my Google Interview.  That's {\it two}
Dogfish Head IPAs you owe me now, SPH.}).  Following is an example of
the bad old style Scheme {\tt cond}, followed by an equivalent example
of Phosphorus' {\tt multiIf}.
\smallskip
\verbatim| 
(cond [(> e 0) (go 'positive e)]
       [(< e 0) (go 'negative e)]
       [t (go 'zero e)])
|endverbatim
\smallskip
% to get vertical bars in \verbatim text, you double them
% The irony of tihs will be lost on anyone but you reading this source code.
\verbatim|
||12 multiIf ||23 gT e 0 32|| ||32 doP e 23||
             ||34 lT e 0 43|| ||43 doN e 34||
             ||56 doN e 65|| 21||
|endverbatim

Of course, the Phosphorous code only becomes more popularizeAble when
dressed up with some comments that look like XML.\footnote{Notice how,
in the Scheme code, square brackets are used to show brackets that
impose structure without being a more usual {\tt (foo bar)} form.  The
same pairings can be moved implicitly into the namedParenthesis of
Phosphorous by using the same digits in different orders to mark
related forms.}

\section Virtual Machine Targets

Most modern programmers are smart enough to figure out that there must
be something better than Java.  Despite its popularity, Java has
become associated with oldness, which is never popular.  These same
programmers are not, however, smart enough to figure out that the Java
Virtual Machine (JVM) is equivalently awful.  This is because they
have to think about and live with the details of the language on a
dayToDay basis, whereas the awfulness of the JVM can hide behind the
thin facade of Eclipse, or in a pinch, the {\tt java} command.  At the
same time, programmerManagers love Java because everybody else is
doing it, which means there's one less thing that can be held against
them when their project finally collapses {\it under the crushing
weight of their foolish incompetence}.  This confluence of forces has
given rise to a number of programming languages which target the JVM.

The interesting observation here is that because the Java programming
language and Java Virtual Machine are (surprise!) so tightlyCoupled,
new language designers are compelled to make their languages such that
they use only those features they can implement efficiently on the
JVM.  For example, implementations of Scheme for the JVM either lack
{\tt call/cc} or have a very slow and slightly buggy implementation of
it.  We call this the {\it Gosling Tarpit}.

Phosphorous escapes the Gosling Tarpit by means of a VirtualVirtual
Machine (V2M).  Some legitimately smart person can then figure out how
to do continuations on the JVM once and for all and we can package it
up into a callWithCurrentContinuation opcode on the V2M.  Retargeting
the V2M to the CLR is a work in progress.\footnote{It would go a lot
faster if we had some research grants.  I'm looking at you SPJ.}

\section Growing a Language

Most new soCalled ``scripting'' programming languages grow over time,
developing new features and incompatibilities fearlessly.  In
contrast, Lisp has been quite slow to regenerate its various language
standards.  As a result, we have designed into Phosphorous a
sophisticated plan for a pattern of growth.  We call it our twoPronged
approach to growth.

One, we make our implementation open source.  We will make our money
by having our language become popular enough that everyone feels
compelled to buy the O'Reilly book for it, and we will write the
O'Reilly book.

Two, we will undergo a vigorous ISO standardization effort.

\section No BoltOns

The worst thing that can be said about a programming language is that
its object system is ``bolted on.''  Let me tell you, our object
system is not bolted on.  It's not even close to bolted on.

\section Lexical Scoping

All modern popular programming languages got lexical scoping wrong the
first few times they tried it\footnote{It is possible that this is why
Emacs Lisp has wisely never made the attempt.}, causing unimaginable
amounts of old code to break when the problems were fixed, as well as
other ills.  Phosphorous eliminates this problem by getting lexical
scoping exactly right in its written material, but introducing the
usual argList bugs in the actual implementation.  In this way, we can
reap all the benefits of being like popular languages, yet our
reputations as language designers can stay pure.  We will leave the
bugs in until the language takes root as a popular alternative to
others.

% Ending with \singlecolumn ballances the two columns to the same
% depth on the last page.
\singlecolumn

\bye
